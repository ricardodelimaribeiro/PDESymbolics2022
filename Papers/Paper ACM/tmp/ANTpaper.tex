\documentclass{amsart}
\usepackage{color}
\begin{document}
\section{Basic Text Statistics}
 \[\begin{array}{ll}
 \text{Language = } & \fbox{$\text{English}$} \\
 \text{Sentiment = } & \text{Neutral} \\
 \text{$\#$ Characters = } & 18321 \\
 \text{$\#$ Words = } & 3367 \\
 \text{$\#$ Sentences = } & 242 \\
 \text{$\#$ Words/Sentence = } & 13.9132 \\
 \text{$\#$ Distinct words = } & 636 \\
 \text{$\#$ Distinct words (without stop words) = } & 522 \\
 \text{Automated Readibility index = } & 11.1553 \\
 \text{Coleman-Liau index = } & 14.0676 \\
 \text{SMOG grade = } & 15.6095 \\
 \text{Glue index = } & 0.467181 \\
 \text{Word repetition index = } & 0.793288 \\
\end{array}\]

\section{Sentence Analysis}

\noindent {\color{green} ********************}


\noindent {\color{red}  {\color{blue} (Long sentence) }  {\color{blue} (Low readability) } 

\noindent [Detection of conserved quantities for PDE schemes]Algorithmic detection of conserved quantities of finite-difference schemes for partial differential equations Diogo A.

\noindent  Gomes King Abdullah University of Science and Technology (KAUST) [0]CEMSE Division [1]AMCS Program Thuwal 23955-6900 Saudi Arabia diogo.

\noindent gomekaust.

\noindent edu.

\noindent sa Friedemann Krannich King Abdullah University of Science and Technology (KAUST) [0]CEMSE Division [1]AMCS Program Thuwal 23955-6900 Saudi Arabia friedemann.

\noindent krannickaust.

\noindent edu.

\noindent sa Ricardo de Lima Ribeiro King Abdullah University of Science and Technology (KAUST) [0]CEMSE Division [1]AMCS Program Thuwal 23955-6900 Saudi Arabia ricardo.

\noindent ribeirkaust.

\noindent edu.

\noindent sa Symbolic computations; Finite-difference schemes; Discrete variational derivative; Discrete partial variational derivative; Conserved quantities; Implicit schemes; Explicit schemes.

\noindent }

\noindent {\color{green} ********************}

\noindent 

\noindent {\color{green} ********************}


\noindent {\color{red}  {\color{blue} (Long sentence) }  {\color{blue} (Low readability) }  {\color{blue} (Style problems: }  {\color{green}  {\bf Hedging and Vague}: like; } {\color{blue} ) } 

\noindent ¡ccs2012¿ ¡concept¿ ¡concept˙id¿10002950.

\noindent 10003714.

\noindent 10003727.

\noindent 10003729¡/concept˙id¿ ¡concept˙desc¿Mathematics of computing Partial differential equations¡/concept˙desc¿ ¡concept˙significance¿500¡/concept˙significance¿ ¡/concept¿ ¡concept¿ ¡concept˙id¿10002950.

\noindent 10003714.

\noindent 10003715.

\noindent 10003750¡/concept˙id¿ ¡concept˙desc¿Mathematics of computing Discretization¡/concept˙desc¿ ¡concept˙significance¿500¡/concept˙significance¿ ¡/concept¿ ¡concept¿ ¡concept˙id¿10002950.

\noindent 10003714.

\noindent 10003715.

\noindent 10003720.

\noindent 10003747¡/concept˙id¿ ¡concept˙desc¿Mathematics of computing Grbner bases and other special bases¡/concept˙desc¿ ¡concept˙significance¿500¡/concept˙significance¿ ¡/concept¿ ¡concept¿ ¡concept˙id¿10010147.

\noindent 10010148.

\noindent 10010149.

\noindent 10010152¡/concept˙id¿ ¡concept˙desc¿Computing methodologies Symbolic calculus algorithms¡/concept˙desc¿ ¡concept˙significance¿500¡/concept˙significance¿ ¡/concept¿ ¡concept¿ ¡concept˙id¿10010147.

\noindent 10010148.

\noindent 10010149.

\noindent 10010155¡/concept˙id¿ ¡concept˙desc¿Computing methodologies Discrete calculus algorithms¡/concept˙desc¿ ¡concept˙significance¿500¡/concept˙significance¿ ¡/concept¿ ¡/ccs2012¿ [500]Mathematics of computing Partial differential equations [500]Mathematics of computing Discretization [500]Mathematics of computing Grbner bases and other special bases [500]Computing methodologies Symbolic calculus algorithms [500]Computing methodologies Discrete calculus algorithms Many partial differential equations (PDEs) admit conserved quantities like mass or energy.

\noindent }

\noindent {\color{green} ********************}

\noindent 

\noindent {\color{green} ********************}


\noindent {\color{red}  {\color{blue} (Style problems: }  {\color{green}  {\bf Hedging and Vague}: often; } {\color{blue} ) } 

\noindent Those quantities are often essential to establish well-posed results.

\noindent }

\noindent {\color{green} ********************}

\noindent 

\noindent {\color{green} ********************}


\noindent {\color{red}  {\color{blue} (Long sentence) } 

\noindent When approximating a PDE by a finite-difference scheme, it is natural to ask whether related discretized quantities remain conserved under the scheme.

\noindent }

\noindent {\color{green} ********************}

\noindent 

\noindent {\color{green} ********************}


\noindent {\color{red}  {\color{blue} (Style problems: }  {\color{green}  {\bf Hedging and Vague}: may; } {\color{blue} ) } 

\noindent Such conservation may establish the stability of the numerical scheme.

\noindent }

\noindent {\color{green} ********************}

\noindent  We present an algorithm for checking the preservation of a polynomial quantity under a polynomial finite-difference scheme.

\noindent  

\noindent {\color{green} ********************}


\noindent {\color{red}  {\color{blue} (Long sentence) } 

\noindent In our algorithm, schemes can be explicit or implicit, have higher-order time and space derivatives, and an arbitrary number of variables.

\noindent }

\noindent {\color{green} ********************}

\noindent  Additionally, we present an algorithm for, given a scheme, finding conserved quantities.

\noindent  

\noindent {\color{green} ********************}


\noindent {\color{red}  {\color{blue} (Low readability) } 

\noindent We illustrate our algorithm by studying several finite-difference schemes.

\noindent }

\noindent {\color{green} ********************}

\noindent 

\noindent {\color{green} ********************}


\noindent {\color{red}  {\color{blue} (Low readability) } 

\noindent The authors were supported by GN01King Abdullah University of Science and Technology (KAUST) baseline funds and GN02KAUST OSR-CRG2021-4674.

\noindent }

\noindent {\color{green} ********************}

\noindent 

\noindent {\color{green} ********************}


\noindent {\color{red}  {\color{blue} (Low readability) }  {\color{blue} (No verb) }

\noindent Introduction.

\noindent }

\noindent {\color{green} ********************}

\noindent  Many partial differential equations (PDEs) admit integral quantities, that are conserved in time.

\noindent   For example, the advection equation preserves energy and the heat equation conserves mass.

\noindent   When approximating PDEs by a finite-difference scheme, the question arises whether such quantities are conserved by the scheme.

\noindent   Such information can be crucial to estimate whether a scheme approximates a PDE accurately and to determine its stability.

\noindent  

\noindent {\color{green} ********************}


\noindent {\color{red}  {\color{blue} (Long sentence) }  {\color{blue} (Style problems: }  {\color{green}  {\bf Hedging and Vague}: often, rather;  {\bf Wordiness}: more; } {\color{blue} ) } 

\noindent While computations for determining conservation are often easy in simple cases, they get rather tedious for more complicated schemes or quantities.

\noindent }

\noindent {\color{green} ********************}

\noindent  Hence, automating such computations in computer algebra systems is desirable.

\noindent  

\noindent {\color{green} ********************}


\noindent {\color{red}  {\color{blue} (Low readability) } 

\noindent An algorithm for finding conserved quantities for (continuous) differential equations was developed and implemented in Maple by Cheviakov (see [1] and [1]).

\noindent }

\noindent {\color{green} ********************}

\noindent 

\noindent {\color{green} ********************}


\noindent {\color{red}  {\color{blue} (No verb) }

\noindent Hereman et al.

\noindent }

\noindent {\color{green} ********************}

\noindent 

\noindent {\color{green} ********************}


\noindent {\color{red}  {\color{blue} (Low readability) } 

\noindent (see for example [1] and [1]), proposed an algorithm to compute conserved densities for semi-discretized schemes for PDEs with first-order time derivative.

\noindent }

\noindent {\color{green} ********************}

\noindent 

\noindent {\color{green} ********************}


\noindent {\color{red}  {\color{blue} (Long sentence) }  {\color{blue} (Low readability) }  {\color{blue} (Style problems: }  {\color{green}  {\bf Wordiness}: which is; } {\color{blue} ) } 

\noindent Their algorithm, which is implemented in a Mathematica package, uses the scaling symmetries of the scheme to construct conserved quantities and calculates their coefficients by using the discrete variational derivative.

\noindent }

\noindent {\color{green} ********************}

\noindent 

\noindent {\color{green} ********************}


\noindent {\color{red}  {\color{blue} (Low readability) } 

\noindent Gao et al.

\noindent  extended this algorithm to first-order time-explicit schemes [1] and to first-order time-implicit schemes [1].

\noindent }

\noindent {\color{green} ********************}

\noindent  They also implemented it in the computer algebra system Reduce.

\noindent   In this paper, we propose an algorithm that checks if a quantity is conserved in time under a scheme.

\noindent  

\noindent {\color{green} ********************}


\noindent {\color{red}  {\color{blue} (Low readability) }  {\color{blue} (Style problems: }  {\color{green}  {\bf Hedging and Vague}: usually; } {\color{blue} ) } 

\noindent Conserved quantities usually involve integral expressions or their discrete analog, sums.

\noindent }

\noindent {\color{green} ********************}

\noindent 

\noindent {\color{green} ********************}


\noindent {\color{red}  {\color{blue} (No verb) }

\noindent Gomes et al.

\noindent }

\noindent {\color{green} ********************}

\noindent  proposed algorithms for the simplification of sums in [1].

\noindent  

\noindent {\color{green} ********************}


\noindent {\color{red}  {\color{blue} (Low readability) } 

\noindent They also developed and implemented algorithms for detecting quantities conserved by PDEs and semi-discretized schemes.

\noindent }

\noindent {\color{green} ********************}

\noindent  We revise these techniques in Section [1].

\noindent  

\noindent {\color{green} ********************}


\noindent {\color{red}  {\color{blue} (Long sentence) }  {\color{blue} (Low readability) } 

\noindent To generalize those methods to situations where we do not sum over all arguments of the functions involved, we introduce the discrete partial variational derivative (Section [1]).

\noindent }

\noindent {\color{green} ********************}

\noindent  The main contribution of this paper is an algorithm for checking the conservation of a quantity under a numerical scheme.

\noindent  

\noindent {\color{green} ********************}


\noindent {\color{red}  {\color{blue} (Long sentence) } 

\noindent Gerdt showed in [1], that if the discrete time derivative of the quantity belongs to the difference ideal generated by the scheme, the quantity is conserved.

\noindent }

\noindent {\color{green} ********************}

\noindent 

\noindent {\color{green} ********************}


\noindent {\color{red}  {\color{blue} (Long sentence) }  {\color{blue} (Style problems: }  {\color{green}  {\bf Hedging and Vague}: may; } {\color{blue} ) } 

\noindent However, some quantities may add to a constant (e.

\noindent g.

\noindent  telescopic sums) and thus be trivially preserved without belonging to the difference ideal.

\noindent }

\noindent {\color{green} ********************}

\noindent 

\noindent {\color{green} ********************}


\noindent {\color{red}  {\color{blue} (Style problems: }  {\color{green}  {\bf Hedging and Vague}: may; } {\color{blue} ) } 

\noindent Moreover, Gerdt’s algorithm may not terminate, as the Grbner basis for the difference ideal may not be finite.

\noindent }

\noindent {\color{green} ********************}

\noindent 

\noindent {\color{green} ********************}


\noindent {\color{red}  {\color{blue} (Long sentence) }  {\color{blue} (Low readability) }  {\color{blue} (Style problems: }  {\color{green}  {\bf Abstract names}: issues; } {\color{blue} ) } 

\noindent We overcome this issues by combining the discrete partial variational derivative with a polynomial ideal (instead of a difference ideal) with finite Grbner basis (Section [1]).

\noindent }

\noindent {\color{green} ********************}

\noindent 

\noindent {\color{green} ********************}


\noindent {\color{red}  {\color{blue} (Long sentence) }  {\color{blue} (Low readability) } 

\noindent Our algorithm works for schemes that are explicit and implicit in time and can treat schemes with several and higher-order time and space derivatives and with several space dimensions.

\noindent }

\noindent {\color{green} ********************}

\noindent  Additionally, we can handle systems of equations and schemes with parameters.

\noindent  

\noindent {\color{green} ********************}


\noindent {\color{red}  {\color{blue} (High-glue sentence) } 

\noindent We have implemented this algorithm as part of a package in Mathematica [1].

\noindent }

\noindent {\color{green} ********************}

\noindent 

\noindent {\color{green} ********************}


\noindent {\color{red}  {\color{blue} (Long sentence) }  {\color{blue} (Low readability) } 

\noindent In the examples, we show that our code finds conserved quantities and proper schemes for the time-implicit and time-explicit discretization of the Burgers equation and a system of PDEs arising in the study of mean-field games (Section [1]).

\noindent }

\noindent {\color{green} ********************}

\noindent 

\noindent {\color{green} ********************}


\noindent {\color{red}  {\color{blue} (Low readability) }  {\color{blue} (No verb) }

\noindent Preliminaries.

\noindent }

\noindent {\color{green} ********************}

\noindent  Throughout this section, we follow the ideas and computations in [1].

\noindent   Here, subscripts denote indices related to coordinates or tuples while superscripts denote indices related to sequences and families.

\noindent   To avoid dealing with boundary terms, we work with periodic functions in [1].

\noindent   Let [1], [1] and [1] be natural numbers.

\noindent   The discrete torus is [1].

\noindent   Define the space [1], extended to [1] by periodicity.

\noindent   The space of functionals [1] (not necessarily linear) on [1] is [1].

\noindent   In the previous definition, the [1] are not necessarily unit vectors and can vary between functionals.

\noindent   Here, we allow for smooth [1].

\noindent   However, later we require [1] to be a polynomial function.

\noindent   Let [1], [1] and define [1].

\noindent   To simplify sums, the discrete variational derivative is a useful tool.

\noindent   Let [1], [1] and [1].

\noindent   Define [1].

\noindent   In [1], we shift the indices due to periodicity of [1] and [1], [1].

\noindent   [1] is the discrete variational derivative of [1].

\noindent  

\noindent {\color{green} ********************}


\noindent {\color{red}  {\color{blue} (Style problems: }  {\color{green}  {\bf Hedging and Vague}: assume; } {\color{blue} ) } 

\noindent Because we assume all related functions to be smooth, the discrete variational derivative always exists.

\noindent }

\noindent {\color{green} ********************}

\noindent  Let us return to Example [1] and compute [1] and hence [1].

\noindent  

\noindent {\color{green} ********************}


\noindent {\color{red}  {\color{blue} (Low readability) }  {\color{blue} (No verb) }

\noindent Fundamental theorem.

\noindent }

\noindent {\color{green} ********************}

\noindent  The algorithms for the simplification of sums presented in [1] rely on the following result: Let [1].

\noindent  

\noindent {\color{green} ********************}


\noindent {\color{red}  {\color{blue} (High-glue sentence) } 

\noindent If [1] for all [1] and if there exists [1] such that [1], then [1] for all [1].

\noindent }

\noindent {\color{green} ********************}

\noindent 

\noindent {\color{green} ********************}


\noindent {\color{red}  {\color{blue} (High-glue sentence) }  {\color{blue} (No verb) }

\noindent Conversely, [1] for all [1], if [1].

\noindent }

\noindent {\color{green} ********************}

\noindent 

\noindent {\color{green} ********************}


\noindent {\color{red}  {\color{blue} (No verb) }

\noindent [1] and hence [1].

\noindent }

\noindent {\color{green} ********************}

\noindent  We use Theorem [1] to examine if different sums represent the same quantity.

\noindent   Let [1] and consider the functionals [1].

\noindent   It is clear, that [1] and [1] represent the same quantity (by shifting [1]).

\noindent   We confirm this, using the discrete variational derivative: [1].

\noindent  

\noindent {\color{green} ********************}


\noindent {\color{red}  {\color{blue} (No verb) }

\noindent Hence, [1] and [1].

\noindent }

\noindent {\color{green} ********************}

\noindent  Therefore, both functionals sum to the same quantity.

\noindent   Let [1].

\noindent  

\noindent {\color{green} ********************}


\noindent {\color{red}  {\color{blue} (High-glue sentence) }  {\color{blue} (Style problems: }  {\color{green}  {\bf Hedging and Vague}: may; } {\color{blue} ) } 

\noindent It may not be obvious, that [1].

\noindent }

\noindent {\color{green} ********************}

\noindent  We confirm this, by computing the discrete variational derivative and noticing that the expression is [1] for [1].

\noindent  

\noindent {\color{green} ********************}


\noindent {\color{red}  {\color{blue} (Low readability) }  {\color{blue} (No verb) }

\noindent The discrete partial variational derivative.

\noindent }

\noindent {\color{green} ********************}

\noindent  Here, we generalize the discrete variational derivative for situations where we keep one (or several) of the arguments of [1] constant.

\noindent   Let [1] be the variables for the functions in [1].

\noindent   We call [1] the space variables.

\noindent   Let [1] and call [1] the time variables.

\noindent   Later in this paper, we only consider the case [1].

\noindent   Let [1], extended to [1] by periodicity in the space variables and let [1].

\noindent   When considering sums, we only sum over the space variables and not over the time variables.

\noindent   The space of functions [1] is [1].

\noindent  

\noindent {\color{green} ********************}


\noindent {\color{red}  {\color{blue} (High-glue sentence) }  {\color{blue} (No verb) }

\noindent Here, [1].

\noindent }

\noindent {\color{green} ********************}

\noindent  Throughout this paper, all quantities, whose conservation in time we check, are functions [1].

\noindent   Define [1].

\noindent   Using the Kronecker delta [1], we rewrite [1].

\noindent   Let [1] and compute [1].

\noindent   We define [1] as the discrete partial variational derivative of [1].

\noindent   Note that [1].

\noindent  

\noindent {\color{green} ********************}


\noindent {\color{red}  {\color{blue} (Low readability) } 

\noindent The discrete partial variational derivative of [1] equals [1], because [1].

\noindent }

\noindent {\color{green} ********************}

\noindent 

\noindent {\color{green} ********************}


\noindent {\color{red}  {\color{blue} (Long sentence) }  {\color{blue} (Low readability) } 

\noindent In our Mathematica implementation, Example [1] is computed by We have a result similar to Theorem [1] for the discrete partial variational derivative: Let [1].

\noindent }

\noindent {\color{green} ********************}

\noindent 

\noindent {\color{green} ********************}


\noindent {\color{red}  {\color{blue} (High-glue sentence) } 

\noindent If [1] for all [1] and if there exists [1] such that [1], then [1] for all [1].

\noindent }

\noindent {\color{green} ********************}

\noindent 

\noindent {\color{green} ********************}


\noindent {\color{red}  {\color{blue} (High-glue sentence) }  {\color{blue} (No verb) }

\noindent Conversely, [1] for all [1], if [1].

\noindent }

\noindent {\color{green} ********************}

\noindent  The proof is similar to the proof of Theorem [1].

\noindent   Difference schemes and algebra.

\noindent  

\noindent {\color{green} ********************}


\noindent {\color{red}  {\color{blue} (Long sentence) } 

\noindent In this section, we define numerical schemes formally and introduce the tools from difference algebra, that we need for our algorithm.

\noindent }

\noindent {\color{green} ********************}

\noindent 

\noindent {\color{green} ********************}


\noindent {\color{red}  {\color{blue} (High-glue sentence) }  {\color{blue} (Style problems: }  {\color{green}  {\bf Hedging and Vague}: assume; } {\color{blue} ) } 

\noindent During the remainder of this paper, we assume that there is only one time variable [1] ([1]).

\noindent }

\noindent {\color{green} ********************}

\noindent  The space of schemes is [1].

\noindent   A scheme is a set of functions [1] that represent the equations [1] for [1], holding pointwise for all points in [1].

\noindent   If a scheme contains a single function [1], we call [1] the scheme.

\noindent  

\noindent {\color{green} ********************}


\noindent {\color{red}  {\color{blue} (Style problems: }  {\color{green}  {\bf Hedging and Vague}: sometimes; } {\color{blue} ) } 

\noindent Sometimes, we also call the expression [1] the scheme.

\noindent }

\noindent {\color{green} ********************}

\noindent 

\noindent {\color{green} ********************}


\noindent {\color{red}  {\color{blue} (Long sentence) } 

\noindent In this work, we only consider finite-difference schemes with fixed step size [1], but one can generalize our results and algorithms to other step sizes.

\noindent }

\noindent {\color{green} ********************}

\noindent 

\noindent {\color{green} ********************}


\noindent {\color{red}  {\color{blue} (Long sentence) } 

\noindent A scheme [1] given by [1] is in time-explicit form if we can rewrite the previous equation (eventually translating the scheme) as [1] for [1] and [1] with [1].

\noindent }

\noindent {\color{green} ********************}

\noindent 

\noindent {\color{green} ********************}


\noindent {\color{red}  {\color{blue} (High-glue sentence) } 

\noindent We call [1] the right-hand side of [1].

\noindent }

\noindent {\color{green} ********************}

\noindent  Consider the forward-difference scheme for the heat equation [1] [1].

\noindent   This scheme is in time-explicit form with right-hand side [1].

\noindent  

\noindent {\color{green} ********************}


\noindent {\color{red}  {\color{blue} (No verb) }

\noindent Difference ideals.

\noindent }

\noindent {\color{green} ********************}

\noindent  Following Gerdt [1], we now introduce difference ideals and how we use them in our algorithm.

\noindent   Let [1].

\noindent   The shift in the [1]-th coordinate for [1] by [1] is [1] understanding that [1] shifts the time variable.

\noindent  

\noindent {\color{green} ********************}


\noindent {\color{red}  {\color{blue} (Style problems: }  {\color{green}  {\bf Hedging and Vague}: possible; } {\color{blue} ) } 

\noindent The set of all possible shifts [1] is [1].

\noindent }

\noindent {\color{green} ********************}

\noindent  Now we construct the difference ideal containing the scheme: Let [1] be the field generated by [1] and the variables [1].

\noindent   Let [1] be the polynomial ring over the field [1] and the variables [1] for [1] and [1].

\noindent   A set [1] is a difference ideal if [1] implies [1], [1] implies [1], [1] implies [1].

\noindent   [1] is the smallest difference ideal containing the scheme [1].

\noindent   A solution of a numerical scheme is a function [1], that makes all translations of the scheme vanish.

\noindent   Hence, every element of the difference ideal generated by the scheme vanishes under [1].

\noindent   In particular, Algorithm [1] seeks to determine if the discrete time derivative [1] belongs to the ideal [1].

\noindent   To examine, if [1], Gerdt proposes the notion of a standard (Grbner) basis for the ideal [1].

\noindent  

\noindent {\color{green} ********************}


\noindent {\color{red}  {\color{blue} (Style problems: }  {\color{green}  {\bf Hedging and Vague}: may; } {\color{blue} ) } 

\noindent This standard basis may not be finite.

\noindent }

\noindent {\color{green} ********************}

\noindent 

\noindent {\color{green} ********************}


\noindent {\color{red}  {\color{blue} (Long sentence) }  {\color{blue} (Low readability) } 

\noindent We overcome this problem by considering polynomial instead of difference algebra and using a smaller polynomial ideal, that is contained in the difference ideal and admits a finite standard basis.

\noindent }

\noindent {\color{green} ********************}

\noindent 

\noindent {\color{green} ********************}


\noindent {\color{red}  {\color{blue} (Long sentence) }  {\color{blue} (Style problems: }  {\color{green}  {\bf Hedging and Vague}: may; } {\color{blue} ) } 

\noindent However, there may be functions [1], whose sums add to zero, even though they may fail to belong to the difference ideal.

\noindent }

\noindent {\color{green} ********************}

\noindent  Hence, we combine the discrete partial variational derivative with polynomial ideals.

\noindent   The time-explicit case.

\noindent  

\noindent {\color{green} ********************}


\noindent {\color{red}  {\color{blue} (Long sentence) } 

\noindent In this section, we present Algorithm [1] for checking the conservation of a quantity under the right-hand side of a scheme in time-explicit form.

\noindent }

\noindent {\color{green} ********************}

\noindent  Our algorithm for general schemes can be found in the subsequent section.

\noindent   Note, that the result of step 2 exists only at time [1] because all instances of [1] have been replaced.

\noindent   Thus, we do not need any computations of difference ideals or Grbner basis.

\noindent   We check for conservation of [1] under the scheme from Example [1].

\noindent   The discrete time derivative (step 1) is [1].

\noindent   Replacing [1] by the right-hand side of the scheme (step 2) gives [1].

\noindent   Then, we compute the discrete partial variational derivative (step 3), which equals zero.

\noindent   Hence (step 4), we have conservation.

\noindent   We verify the result from Example [1], using our implementation in Mathematica of Algorithm [1].

\noindent   The general case.

\noindent   In this section, we present Algorithm [1] that deals with general, not necessarily time-explicit, schemes.

\noindent   We explain its steps in detail below and demonstrate the algorithm in Example [1].

\noindent   Step 1: Build the discrete time derivative.

\noindent   We build the discrete time derivative by subtracting [1] from [1].

\noindent  

\noindent {\color{green} ********************}


\noindent {\color{red}  {\color{blue} (High-glue sentence) } 

\noindent This task is done in our code by the TimeDifference.

\noindent }

\noindent {\color{green} ********************}

\noindent  The discrete time derivative for [1] is Step 2: Translate the scheme.

\noindent  

\noindent {\color{green} ********************}


\noindent {\color{red}  {\color{blue} (Long sentence) }  {\color{blue} (Low readability) } 

\noindent To make the step from difference to polynomial algebra to guarantee termination of the algorithm, we use a standard idea in symbolic computations and consider every instance of [1] as a variable of a multivariate polynomial.

\noindent }

\noindent {\color{green} ********************}

\noindent 

\noindent {\color{green} ********************}


\noindent {\color{red}  {\color{blue} (Long sentence) } 

\noindent Hence, we compute all translations of the scheme, such that all variables in the polynomial associated with [1] appear in the polynomials associated with the translated scheme.

\noindent }

\noindent {\color{green} ********************}

\noindent  The algorithm treats the discrete time derivative [1] as the polynomial equation [1].

\noindent  

\noindent {\color{green} ********************}


\noindent {\color{red}  {\color{blue} (High-glue sentence) } 

\noindent Let [1] and [1] with [1].

\noindent }

\noindent {\color{green} ********************}

\noindent  The stencil of [1] is the [1]-tuple of sets of vectors [1].

\noindent  

\noindent {\color{green} ********************}


\noindent {\color{red}  {\color{blue} (High-glue sentence) } 

\noindent The range of the stencil of [1] is [1], where [1], [1], and [1].

\noindent }

\noindent {\color{green} ********************}

\noindent  Here, [1] denotes the discrete interval in [1], i.

\noindent e.

\noindent  [1].

\noindent  

\noindent {\color{green} ********************}


\noindent {\color{red}  {\color{blue} (Long sentence) } 

\noindent We calculate the minimal necessary translations of the scheme, such that all instances of [1], that appear in the discrete time derivative, also appear in the translated scheme.

\noindent }

\noindent {\color{green} ********************}

\noindent 

\noindent {\color{green} ********************}


\noindent {\color{red}  {\color{blue} (Long sentence) } 

\noindent This is done by elementwise subtracting the range of the stencil of the scheme from the range of the stencil of the discrete time derivative.

\noindent }

\noindent {\color{green} ********************}

\noindent  We denote the resulting translated system by [1].

\noindent  

\noindent {\color{green} ********************}


\noindent {\color{red}  {\color{blue} (High-glue sentence) } 

\noindent If for any [1] involved [1], we write [1] with the convention that [1].

\noindent }

\noindent {\color{green} ********************}

\noindent  Further, [1] in the translations results in the use of the respective entry of the original scheme without translations.

\noindent   Consider the discrete time derivative [1] and the scheme [1].

\noindent   Then, the stencil of the discrete time derivative equals [1] with range [1].

\noindent   The stencil of the scheme equals the set of stencils of the equations in the scheme, i.

\noindent e.

\noindent  [1] with range [1].

\noindent  

\noindent {\color{green} ********************}


\noindent {\color{red}  {\color{blue} (Long sentence) }  {\color{blue} (High-glue sentence) } 

\noindent Hence, we get the translations for the first equation of the scheme by [1] and for the second equation of the scheme by [1].

\noindent }

\noindent {\color{green} ********************}

\noindent 

\noindent {\color{green} ********************}


\noindent {\color{red}  {\color{blue} (Long sentence) }  {\color{blue} (High-glue sentence) } 

\noindent Therefore, we translate the first equation of the scheme by [1] and [1] and the second one by [1] to get the translated scheme [1].

\noindent }

\noindent {\color{green} ********************}

\noindent  Step 3 and 4: Compute the Grbner basis and reduce the discrete time derivative.

\noindent   Polynomial ideals and Grbner bases.

\noindent  

\noindent {\color{green} ********************}


\noindent {\color{red}  {\color{blue} (Long sentence) } 

\noindent We found a finite set of polynomials (the translated scheme and the discrete time derivative) with a finite number of instances of [1] in the previous step.

\noindent }

\noindent {\color{green} ********************}

\noindent  Now we reduce [1], using the translated scheme.

\noindent   Here, we adapt the definitions and theorems from [1] to our setting.

\noindent   Let [1] be as in Definition [1].

\noindent   Let [1] be the polynomial ring generated by [1] and all instances of [1] that occur in [1].

\noindent   We call a set [1] a (polynomial) ideal if [1] implies [1], [1] implies [1].

\noindent   We denote by [1] the smallest (polynomial) ideal containing [1].

\noindent  

\noindent {\color{green} ********************}


\noindent {\color{red}  {\color{blue} (Long sentence) } 

\noindent Given the ideal [1] and the discrete time derivative [1], we want to determine if [1] or if we can write [1] in a simpler form, using [1].

\noindent }

\noindent {\color{green} ********************}

\noindent  Hence, using multivariate polynomial division, we search for [1] such that [1].

\noindent  

\noindent {\color{green} ********************}


\noindent {\color{red}  {\color{blue} (Style problems: }  {\color{green}  {\bf Hedging and Vague}: may;  {\bf Wordiness}: unique; } {\color{blue} ) } 

\noindent Unfortunately, the resulting remainder [1] may not be unique [page 14, Example 1.

\noindent 2.

\noindent 3]hibi13, i.

\noindent e.

\noindent  non-zero, although [1] belongs to [1].

\noindent }

\noindent {\color{green} ********************}

\noindent  Replacing [1] by a Grbner basis for the ideal [1] guarantees the uniqueness of the remainder of the polynomial division.

\noindent  

\noindent {\color{green} ********************}


\noindent {\color{red}  {\color{blue} (Long sentence) } 

\noindent Contrary to the standard basis for the difference ideal that Gerdt’s algorithm computes, the Grbner basis for the polynomial ideal always exists and is finite.

\noindent }

\noindent {\color{green} ********************}

\noindent 

\noindent {\color{green} ********************}


\noindent {\color{red}  {\color{blue} (Long sentence) } 

\noindent A Grbner basis is defined up to the order of the monomials involved, so, depending on the order, we get different remainders of the polynomial division.

\noindent }

\noindent {\color{green} ********************}

\noindent 

\noindent {\color{green} ********************}


\noindent {\color{red}  {\color{blue} (Low readability) }  {\color{blue} (No verb) }

\noindent Polynomial reduction.

\noindent }

\noindent {\color{green} ********************}

\noindent  In Algorithm [1], we consider two monomial orders: (a) lexicographic and (b) explicit.

\noindent  

\noindent {\color{green} ********************}


\noindent {\color{red}  {\color{blue} (Long sentence) } 

\noindent To reduce [1], we compute the Grbner basis of [1] with respect to the monomial order and then calculate the remainder of the polynomial division of [1] with respect to this Grbner basis.

\noindent }

\noindent {\color{green} ********************}

\noindent 

\noindent {\color{green} ********************}


\noindent {\color{red}  {\color{blue} (Long sentence) } 

\noindent The explicit monomial order (b) results in the Grbner basis that eliminates the instances of [1] at the latest time (with the largest [1]).

\noindent }

\noindent {\color{green} ********************}

\noindent 

\noindent {\color{green} ********************}


\noindent {\color{red}  {\color{blue} (Style problems: }  {\color{green}  {\bf Hedging and Vague}: may, might, possible; } {\color{blue} ) } 

\noindent This elimination might not always be possible, hence the reduction may leave the discrete time derivative unchanged.

\noindent }

\noindent {\color{green} ********************}

\noindent 

\noindent {\color{green} ********************}


\noindent {\color{red}  {\color{blue} (Long sentence) } 

\noindent Then we repeat this process for the instance at the second latest time and continue until all instances of [1] have been eliminated.

\noindent }

\noindent {\color{green} ********************}

\noindent 

\noindent {\color{green} ********************}


\noindent {\color{red}  {\color{blue} (Long sentence) } 

\noindent Out of the two remainders from (a) and (b), we choose the result with the least number of different instances of time.

\noindent }

\noindent {\color{green} ********************}

\noindent  Our code also supports other elimination orders, including user-defined ones.

\noindent   Step 5 to 7: Compute the discrete partial variational derivative and reduce again.

\noindent   In the next step, we take the resulting expression and compute its discrete partial variational derivative.

\noindent   Then, we repeat Step 2 to Step 4 applied to the discrete partial variational derivative.

\noindent   To demonstrate Algorithm [1], we use the setting from Example [1].

\noindent   We compute the discrete time derivative (step 1) [1].

\noindent   For the translations (step 2), the stencil of the scheme is [1] with range [1].

\noindent   The discrete time derivative has stencil [1] and range [1].

\noindent  

\noindent {\color{green} ********************}


\noindent {\color{red}  {\color{blue} (Long sentence) } 

\noindent The translations are [1] which results in no translations, as the range of the stencil of the scheme is greater than the range of the stencil of the discrete time derivative.

\noindent }

\noindent {\color{green} ********************}

\noindent  Hence, the Grbner basis (step 3) coincides with the original scheme for both monomial orders.

\noindent   The reduction using the lexicographic monomial order yields as remainder [1].

\noindent   For the reduction using the elimination order, we first eliminate [1] and then the remaining instances of [1].

\noindent   The elimination order yields the same remainder as the lexicographic order.

\noindent   Hence, we do not need to check for the number of instances of time (step 4).

\noindent   We calculate the discrete partial variational derivative (step 5) of [1] that equals [1].

\noindent   Hence, repeating the above procedure (step 6) becomes unnecessary.

\noindent   Therefore, (step 7) we see that the scheme conserves the quantity.

\noindent  

\noindent {\color{green} ********************}


\noindent {\color{red}  {\color{blue} (Long sentence) } 

\noindent We verify our result from Example [1], using DiscreteConservedQ, which detects automatically if we are in the time-explicit or the general setting.

\noindent }

\noindent {\color{green} ********************}

\noindent  The algorithm can only detect conservation, but can not check if something is not conserved for sure.

\noindent  

\noindent {\color{green} ********************}


\noindent {\color{red}  {\color{blue} (Long sentence) }  {\color{blue} (High-glue sentence) }  {\color{blue} (Style problems: }  {\color{green}  {\bf Hedging and Vague}: should; } {\color{blue} ) } 

\noindent So a resulting False should be understood as our algorithm not detecting conservation, not as there being no conservation at all.

\noindent }

\noindent {\color{green} ********************}

\noindent  A basis for conserved quantities.

\noindent   So far, we have discussed how to check if a quantity is preserved in time under a scheme.

\noindent   But it is also desirable to have a systematic way to find conserved quantities, given a scheme.

\noindent  

\noindent {\color{green} ********************}


\noindent {\color{red}  {\color{blue} (Long sentence) } 

\noindent Algorithm [1] finds, for a time-explicit scheme, a basis for conserved quantities that are generated by monomials up to a (total) degree.

\noindent }

\noindent {\color{green} ********************}

\noindent 

\noindent {\color{green} ********************}


\noindent {\color{red}  {\color{blue} (Low readability) } 

\noindent We have implemented this algorithm in the FindDiscreteConservedQuantityBasis.

\noindent }

\noindent {\color{green} ********************}

\noindent 

\noindent {\color{green} ********************}


\noindent {\color{red}  {\color{blue} (Long sentence) } 

\noindent We find a basis for conserved quantities that are generated by [1] and [1] and that have at most degree 3, admitted by the scheme for the heat equation from Example [1].

\noindent }

\noindent {\color{green} ********************}

\noindent 

\noindent {\color{green} ********************}


\noindent {\color{red}  {\color{blue} (No verb) }

\noindent Examples and applications.

\noindent }

\noindent {\color{green} ********************}

\noindent  We illustrate our code with the Burgers equation and a system of mean-field games.

\noindent  

\noindent {\color{green} ********************}


\noindent {\color{red}  {\color{blue} (No verb) }

\noindent Burgers equation.

\noindent }

\noindent {\color{green} ********************}

\noindent  The Burgers equation is the PDE [1] [1] in one space dimension.

\noindent   Any function of [1] is a conserved quantity.

\noindent   We are interested in discretizations that preserve mass [1].

\noindent   We check for conservation of mass using a forward-difference discretization: We want to find a scheme that preserves this quantity.

\noindent  

\noindent {\color{green} ********************}


\noindent {\color{red}  {\color{blue} (Long sentence) }  {\color{blue} (Style problems: }  {\color{green}  {\bf Hedging and Vague}: appropriate; } {\color{blue} ) } 

\noindent Therefore we check for conservation under a class of schemes with a parameter, to find an appropriate choice for this parameter.

\noindent }

\noindent {\color{green} ********************}

\noindent  We discretize [1] using a three-point stencil with a parameter [1].

\noindent   We have conservation if we choose [1], corresponding to the standard central difference.

\noindent  

\noindent {\color{green} ********************}


\noindent {\color{red}  {\color{blue} (Long sentence) }  {\color{blue} (Low readability) } 

\noindent So far, we have only seen schemes that are in time-explicit form, but our algorithm allows also for general schemes: We consider the scheme from the previous example but in the time-implicit version.

\noindent }

\noindent {\color{green} ********************}

\noindent  Conserved quantities for a mean-field game.

\noindent  

\noindent {\color{green} ********************}


\noindent {\color{red}  {\color{blue} (Long sentence) } 

\noindent As a second application, we use our code to study conserved quantities admitted by the discretization of a system of PDEs.

\noindent }

\noindent {\color{green} ********************}

\noindent 

\noindent {\color{green} ********************}


\noindent {\color{red}  {\color{blue} (No verb) }

\noindent In [1], Gomes et al.

\noindent }

\noindent {\color{green} ********************}

\noindent  derived the following system of PDEs [1].

\noindent   This system admits the conserved quantities [1] and [1].

\noindent   Because this system was derived from a forward-forward mean-field game, we discretize it forward in time.

\noindent  

\noindent {\color{green} ********************}


\noindent {\color{red}  {\color{blue} (Long sentence) }  {\color{blue} (Low readability) } 

\noindent We check with our code if the scheme admits the same preserved quantities as the continuous system: We reproduce this result by noticing, that the scheme is in time-explicit form: Those are the only conserved quantities for this scheme, that are polynomials up to degree 4 in [1] and [1].

\noindent }

\noindent {\color{green} ********************}

\noindent 

\noindent {\color{green} ********************}


\noindent {\color{red}  {\color{blue} (Low readability) } 

\noindent The related backward-forward system reads [1].

\noindent }

\noindent {\color{green} ********************}

\noindent 

\noindent {\color{green} ********************}


\noindent {\color{red}  {\color{blue} (Long sentence) } 

\noindent This system admits the same conserved quantities as the forward-forward system, but we approximate it explicitly (forward) in time for [1] and implicitly (backward) for [1].

\noindent }

\noindent {\color{green} ********************}

\noindent 

\noindent {\color{green} ********************}


\noindent {\color{red}  {\color{blue} (Long sentence) }  {\color{blue} (Low readability) }  {\color{blue} (Style problems: }  {\color{green}  {\bf Hedging and Vague}: possible; } {\color{blue} ) } 

\noindent Our code can handle this setting, if we specify an order for the elimination of the variables, telling the code to use an explicit monomial order for [1] and an implicit one for [1]: Possible extensions and concluding remarks.

\noindent }

\noindent {\color{green} ********************}

\noindent 

\noindent {\color{green} ********************}


\noindent {\color{red}  {\color{blue} (Low readability) }  {\color{blue} (No verb) }

\noindent Polynomial treatment of non-polynomial expressions.

\noindent }

\noindent {\color{green} ********************}

\noindent 

\noindent {\color{green} ********************}


\noindent {\color{red}  {\color{blue} (Style problems: }  {\color{green}  {\bf Hedging and Vague}: may, may be possible, possible; } {\color{blue} ) } 

\noindent It may be possible to extend our methods to non-polynomial PDEs and schemes.

\noindent }

\noindent {\color{green} ********************}

\noindent  In this case, non-polynomial expressions can be handled by writing them in polynomial form.

\noindent   For example, treat the expression [1] as the polynomial [1].

\noindent  

\noindent {\color{green} ********************}


\noindent {\color{red}  {\color{blue} (Style problems: }  {\color{green}  {\bf Wordiness}: more; } {\color{blue} ) } 

\noindent Translate the scheme more accurately.

\noindent }

\noindent {\color{green} ********************}

\noindent 

\noindent {\color{green} ********************}


\noindent {\color{red}  {\color{blue} (Long sentence) } 

\noindent In our elimination procedure, we work with a specific polynomial ideal that is a subset of the difference ideal generated by the scheme.

\noindent }

\noindent {\color{green} ********************}

\noindent 

\noindent {\color{green} ********************}


\noindent {\color{red}  {\color{blue} (High-glue sentence) }  {\color{blue} (Style problems: }  {\color{green}  {\bf Wordiness}: more; } {\color{blue} ) } 

\noindent However, it could be necessary to translate the scheme more than we did in our algorithm to get cancellation.

\noindent }

\noindent {\color{green} ********************}

\noindent  Therefore, flexible methods to determine the translations of the scheme are desirable.

\noindent   Discrete conserved quantities that are not conserved by the PDE.

\noindent  

\noindent {\color{green} ********************}


\noindent {\color{red}  {\color{blue} (Long sentence) }  {\color{blue} (Style problems: }  {\color{green}  {\bf Hedging and Vague}: may; } {\color{blue} ) } 

\noindent It may arise the question if we can find conserved quantities, that are not preserved by the original PDE.

\noindent  Gerdt et al.

\noindent }

\noindent {\color{green} ********************}

\noindent  ([1],[1]) define the notion of s-consistency, which guarantees that all discrete quantities are also preserved in the continuous setting.

\noindent  

\noindent {\color{green} ********************}


\noindent {\color{red}  {\color{blue} (Style problems: }  {\color{green}  {\bf Hedging and Vague}: possible, would; } {\color{blue} ) } 

\noindent One possible extension of our code would be to check automatically if s-consistency holds for the translated scheme.

\noindent }

\noindent {\color{green} ********************}

\noindent 

\noindent {\color{green} ********************}


\noindent {\color{red}  {\color{blue} (Low readability) } 

\noindent Concluding remarks.

\noindent }

\noindent {\color{green} ********************}

\noindent  We present an algorithm for checking the conservation of a quantity under a finite-difference scheme.

\noindent   Our algorithm allows for systems of equations, arbitrary time and space derivatives, and both explicit and implicit schemes.

\noindent   Also, we implemented a function for finding conserved quantities admitted by a scheme.

\noindent  

\noindent {\color{green} ********************}


\noindent {\color{red}  {\color{blue} (Long sentence) } 

\noindent We use our implementation of the algorithm to analyze the conservation properties of several schemes for PDEs that arise in applications.

\noindent }

\noindent {\color{green} ********************}

\noindent 

\noindent {\color{green} ********************}


\noindent {\color{red}  {\color{blue} (Low readability) } 

\noindent ACM-Reference-Format references}

\noindent {\color{green} ********************}

\noindent \end{document}